% Metódy inžinierskej práce

\documentclass[10pt,twoside,slovak,a4paper]{article}

\usepackage[slovak]{babel}
%\usepackage[T1]{fontenc}
\usepackage[IL2]{fontenc} % lepšia sadzba písmena Ľ než v T1
\usepackage[utf8]{inputenc}
\usepackage{graphicx}
\usepackage{url} % príkaz \url na formátovanie URL
\usepackage{hyperref} % odkazy v texte budú aktívne (pri niektorých triedach dokumentov spôsobuje posun textu)

\usepackage{cite}
%\usepackage{times}

\pagestyle{headings}

\title{Názov\thanks{Semestrálny projekt v predmete Metódy inžinierskej práce, ak. rok 2015/16, vedenie: Meno Priezvisko}} % meno a priezvisko vyučujúceho na cvičeniach

\author{Meno Priezvisko\\[2pt]
	{\small Slovenská technická univerzita v Bratislave}\\
	{\small Fakulta informatiky a informačných technológií}\\
	{\small \texttt{...@stuba.sk}}
	}

\date{\small 30. september 2015} % upravte

\begin{document}

\maketitle

\section{Rámocová téma}

Týmto článkom by som chcel čitateľom ukázať ako sa za posledné roky stalo z tvorby webových stránok hra. Nie asi taká, ako si každý tom týmto pojmom predstaví. Ide o hru s našou predstavivosťou. Máme možnosti a technológie, vďaka ktorým sa môžeme vyhrať s našou webovou stránkou a vytvoriť tak umelecké dielo, hru, alebo len zaujímavú vychytávku pre používateľov.

Nikdy predtým tvorba webových stránok nebola jednoduchšia, dnes sme schopný vytvoriť webovú stránku prakticky bez nutnosti ovládať základne poznanie o webe ako takom, máme aplikácie, ktoré skladajú prvky za nás. Mojim hlavným zameraním bude konkrétne 3D. Od začiatkov ako sa nám zobrazuje až po tvorbu 3D hier, interakcii v aktuálnom období.

% inak svieti error
\cite{Kang:FODA}

% týmto sa generuje zoznam literatúry z obsahu súboru literatura.bib podľa toho, na čo sa v článku odkazujete
\bibliography{literatura}
\bibliographystyle{plain} % prípadne alpha, abbrv alebo hociktorý iný
\end{document}

